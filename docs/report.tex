
%%%%%%%%%%%%%%%%%%%%%%% file typeinst.tex %%%%%%%%%%%%%%%%%%%%%%%%%
%
% This is the LaTeX source for the instructions to authors using
% the LaTeX document class 'llncs.cls' for contributions to
% the Lecture Notes in Computer Sciences series.
% http://www.springer.com/lncs       Springer Heidelberg 2006/05/04
%
% It may be used as a template for your own input - copy it
% to a new file with a new name and use it as the basis
% for your article.
%
% NB: the document class 'llncs' has its own and detailed documentation, see
% ftp://ftp.springer.de/data/pubftp/pub/tex/latex/llncs/latex2e/llncsdoc.pdf
%
%%%%%%%%%%%%%%%%%%%%%%%%%%%%%%%%%%%%%%%%%%%%%%%%%%%%%%%%%%%%%%%%%%%


\documentclass[runningheads,a4paper]{llncs}

\usepackage{amssymb}
\setcounter{tocdepth}{3}
\usepackage{graphicx}

\usepackage{url}
\urldef{\mailsa}\path|{alfred.hofmann, ursula.barth, ingrid.haas, frank.holzwarth,|
\urldef{\mailsb}\path|anna.kramer, leonie.kunz, christine.reiss, nicole.sator,|
\urldef{\mailsc}\path|erika.siebert-cole, peter.strasser, lncs}@springer.com|    
\newcommand{\keywords}[1]{\par\addvspace\baselineskip
\noindent\keywordname\enspace\ignorespaces#1}

\begin{document}

\mainmatter  % start of an individual contribution

% first the title is needed
\title{Sistema de Recuperación de Información}

% a short form should be given in case it is too long for the running head
\titlerunning{Sistema de Recuperación de Información}

% the name(s) of the author(s) follow(s) next
%
% NB: Chinese authors should write their first names(s) in front of
% their surnames. This ensures that the names appear correctly in
% the running heads and the author index.
%
\author{Carlos Rafael Ortega Lezacano \and Eric Martín García \\ Grupo C511}
%
\authorrunning{Sistema de Recuperación de Información}
% (feature abused for this document to repeat the title also on left hand pages)

% the affiliations are given next; don't give your e-mail address
% unless you accept that it will be published
\institute{Universidad de la Habana}

%
% NB: a more complex sample for affiliations and the mapping to the
% corresponding authors can be found in the file "llncs.dem"
% (search for the string "\mainmatter" where a contribution starts).
% "llncs.dem" accompanies the document class "llncs.cls".
%

\toctitle{Sistema de Recuperación de Información}
\tocauthor{Authors' Instructions}
\maketitle


\begin{abstract}
The abstract should summarize the contents of the paper and should
contain at least 70 and at most 150 words. It should be written using the
\emph{abstract} environment.
\keywords{We would like to encourage you to list your keywords within
the abstract section}
\end{abstract}


\section{Introducción}

\section{Diseño del Sistema}

Un sistema de recuperación de información (IRS) se compone por el cuadruplo $<D,\ Q,\ F,\ R(q_j, d_j)>$, donde $D$ es un conjunto de representaciones de los documentos, $Q$ es un conjunto compuesto por representaciones lógicas de los pedidos que el usuario realiza al sistema, $F$ es un framework para modelar las representaciones y $R$ es una función de orden donde a cada consulta $q_j$ y un documento $d_j$ le asigna un valor acorde a la relevancia del documento para esa consulta. El proceso de interacción de un sistema con un usuario sigue el siguiente comportamiento: Primeramente el usuario realiza una consulta (un elemento de $Q$), esta pasa al motor de búsqueda, este es la componente más importante del sistema ya que en este se realizan las representaciones de los documentos\footnote{En nuestro caso es un IRS vectorial por tanto la representación interna se realiza mediante vectores cuyas componentes son pesos asociados a la entidad resultante del preprocesamiento}, además de representar la consulta en formato interno, el sistema dará como resultado una lista ordenada por relevancia de los documentos del corpus. Analizaremos a continuación el procesador de consultas, como se compone el motor de búsqueda y finalmente la salida del sistema. En nuestro caso el sistema va enfocado en la búsqueda de documentos, un documento tiene varias características, dos de las más comunes y útiles a la hora de realizar búsquedas es el título y el cuerpo del documento.

\subsection*{Motor de Búsqueda}

El motor de búsqueda contiene el modelo de recuperación de información, además en este se realiza el preprocesado de los documentos y su representación interna, también cada consulta se almacena en su forma interna para ser empleada en otras tareas. Además contiene la función de ranking la cual clasifica la relevancia de los documentos del corpus acorde a la consulta y el IRS.

\subsubsection{Preprocesamiento:} 

Para representar los documentos primeramente es necesario realizar un preprocesado, en nuestro sistema procedemos de forma simple empleando recursos del procesamiento de lenguaje natural. Las siguientes transformaciones son realizadas para cada documento del corpus:

\begin{enumerate}
	\item \textbf{Tokenizar el documento:} Primeramente es necesario crear la representación básica de un documento, la división en tokens, estos son unidades básicas, de esta forma podemos realizar procesamiento más avanzado sobre el texto, ya desde esta parte del procesamiento podemos llevar a minúscula todas las palabras, eliminar símbolos y signos de puntuación. También es posible convertir los números a texto, de esta forma no tendremos problemas a la hora de construir los diversos modelos. \\
	\item \textbf{Eliminación de \textit{stopwords}:} Las \textit{stopwords} como sabemos son palabras vacias de significado, las cuales pueden servir de nexo entre entidades o funcionan como modificadores. Si incluimos estas palabras en la representación del documento se afecta la efectividad del modelo debido a la alta frecuencia que poseen estas palabras en los textos. \\
	\item \textbf{Stemming:} Este método busca relacionar palabras con igual significado
	pero que difieren en cuanto a la escritura, por la aplicación de prefijos o
	sufijos, se encuentran en plural o singular entre otras diferencias, en este caso no usaremos lematización que es común que se emplee luego del stemming, con remover las partes correspondientes es suficiente para obtener buenos resultados.
\end{enumerate} 

Estas técnicas son las empleadas por el sistema para obtener un conjunto de textos formados por entidades con un significado y peso correcto para empezar a construir el modelo, pero antes de definir el modelo es necesario definir como serán estas entidades y que representación interna tendrán. Se eligió emplear los tokens como entidad del sistema, de esta forma se construye una colección con el siguiente formato:  \\

\noindent
{\it Representación del titulo y el cuerpo del documento preprocesado}
\begin{verbatim}
	    corpus: {
                   [id: n,
                   'title': Tokenize(doc_title),
                   'body': Tokenize(doc_body),
                   ],
                   ...
             }
\end{verbatim}
%
\noindent

\subsubsection{Vectorización:} De esta forma contamos con el identificador para cada documento del corpus, su titulo y cuerpo preprocesado. Pero todavia tenemos tokens los cuales no pueden ser la entrada para el modelo, es necesario realizar una vectorización
empleando TF-IDF. Para calcular los valores de frecuencia emplearemos el titulo y el cuerpo del documento juntos, sin realizar distinción, o sea el titulo podría ser una oración más del texto.

\paragraph*{TF-IDF:} Es una técnica para cuantificar una palabra en documentos, generalmente calculamos un peso para cada palabra que significa la importancia de la palabra en el documento y corpus. Este método es una técnica ampliamente utilizada en recuperación de información y minería de textos. Esta se representa por $w(t_i, d_j)$ donde $t_i$ es un término del documento $d_j$, en nuestro caso un los $t_j$ son tokens. \\

El TF-IDF se compone de dos conceptos fundamentales que se apoyan en el procesamiento de los documentos, el TF (Frecuencia de Termino) y el IDF (Frecuencia Inversa de Documento).

\paragraph*{TF:} Mide la frecuencia de una palabra en un documento. Esto depende en gran medida de la longitud del documento y de la generalidad de la palabra, por ejemplo, una palabra muy común como "casa" puede aparecer varias veces en un documento. Como los documentos pueden tener una longitud variable entonces contar simplemente las palabras podría dar prioridad a documentos de mayor tamaño, por eso es que se divide la cantidad de ocurrencias por el número total de palabra del documento $N(d_j)$. TF es individual para cada documento y palabra, por lo tanto, podemos formular TF de la siguiente manera:

\begin{equation}
	f_s(t_i, d_j) = \frac{C(t_i, d_j)}{N(d_j)}
\end{equation}

Donde $C(t_i, d_j)$ representará la cantidad de ocurrencias del termino en el documento. Recordemos que estamos vectorizando los documentos por tanto es necesario introducir el concepto de \textbf{vocabulario}, este es un conjunto formado por todas las palabras del corpus, o sea todas las palabras que hay en cada documento, de esta forma garantizamos que los vectores para cada documento tengan siempre las misma dimensión, donde cada componente tendrá el valor de frecuencia asociado para esa palabra, este estará en el intervalo $[0,1]$. Aunque empleando esta frecuencia es posible obtener resultados, todavía podemos mejorar los valores para el vector. Antes de definir IDF es necesario definir Frecuencia de Documento, ya que su inversa es IDF.

\paragraph*{DF:} Esto mide la importancia del documento en todo el conjunto de corpus, esto es muy similar a TF. La única diferencia es que TF es el contador de frecuencia para un término t en el documento d, donde DF es el recuento de apariciones del término $t_i$ en el conjunto de documentos $D$. En otras palabras, DF es el número de documentos en los que está presente la palabra. Consideramos una ocurrencia si el término consta en el documento al menos una vez, no necesitamos saber el número de veces que el término está presente. 

Como en el caso de TF también normalizamos dividiendo por el número total de documentos. Nuestro principal objetivo es conocer la informatividad de un término, para eso invertimos DF.

\paragraph*{IDF:} Es la inversa de la frecuencia del documento que mide la informatividad del término t. Cuando calculemos el IDF, será muy bajo para las palabras más frecuentes. Como es posible que el DF sea cero entonces es necesario realizar algunas modificaciones para no dividir entre 0, además si el corpus es grande podemos entonces obtener un valor que no sea de utilidad por lo tanto emplearemos para calcular el IDF.

\begin{equation}
	\overline{d}_s(t_i) = \log{\frac{|D|}{d_s(t_i) + 1}}
\end{equation}

Ahora combinemos ambas frecuencias, de esta forma tendremos una relación entre términos de mucha ocurrencia con respecto a aquellos que de baja frecuencia en los documentos.

\begin{equation}
	w(t_i, d_j) = f_s(t_i, d_j) \cdot \log{\frac{|D|}{d_s(t_i) + 1}}
\end{equation}

Hasta el momento se ha expuesto el concepto de TF-IDF para documentos pero ignorando si tienen titulo o no. Los textos que conforman el corpus del motor de búsqueda tienen diversas características y hemos elegido el titulo y el cuerpo para representar, por lo tanto debemos determinar cual de estos es más importante, o sea para dar un valor de relevancia a un documento que tiene mayor peso el título o el cuerpo, para ello definamos el parámetro $\alpha$ el cual define cuanto peso le damos al título a la hora de calcular la relevancia, de esta forma tendríamos que determinar los valores de TF-IDF tanto para el título como para el cuerpo, relacionando estos valores con $\alpha$ para obtener la frecuencia final para el término en el documento. Como los valores de TF-IDF para un término no toman en cuenta si este se encontraba en el titulo del documento o en el cuerpo, como se expreso anteriormente entonces definamos $w_t$ como el valor de TF-IDF para un término en el título y $w_b$ para cuando se encuentra en el cuerpo del documento. Finalmente obtenemos la expresión.

\begin{equation}
	w(t_i, d_j) = w_t(t_i, d_j) · \alpha + w_b(t_i, d_j) · (1 - \alpha), \qquad \alpha \in [0, 1]
\end{equation}

Podemos notar como el valor de $\alpha$ es un coeficiente asociado a las frecuencias de cada parte del documento, además un aspecto importante a tener en cuenta es lo siguiente: Si el termino $t_i$ aparece tanto en el título como en el cuerpo del documento entonces $w_t(t_i, d_j) = w_b(t_i, d_j)$, debido a que no se tomó en cuenta la posición a la hora de calcular los valores de frecuencia, por lo tanto el valor final sería: 

\begin{equation}
	w(t_i, d_j) = w_b(t_i, d_j)
\end{equation}

Ya tenemos el valor de frecuencia además de que empleamos el parámetro  $\alpha$ para definir la importancia que tiene el título o no al momento de determinar la relevancia, pasemos entonces a vectorizar los documentos. Para vectorizar el documento emplearemos un Bag of Words.

\paragraph*{Bag of Words:} Este consiste en tomar todas las palabras del corpus que constituyen el vocabulario ($V$) y formar una colección donde cada palabra tiene asociado un índice, de esta forma obtenemos un vector $\overrightarrow{d_j}$ de dimension $1 \times |V|$ asociado al documento $d_j$, donde en cada componente se encuentra el valor de TF-IDF asociado a la palabra.

De esta forma obtenemos para cada documento del corpus el vector $\overrightarrow{d_j}$, ahora pasemos a determinar como se procesa la consulta para obtener su vector correspondiente y dependiendo de este que función de ranking se empleó.

\subsection*{Consultas}

Como sabemos en el sistema existe un conjunto $Q$ que contiene las consultas realizadas por el usuario, una consulta $q$ no es más que una oración u entidad empleada para realizar la búsqueda. Para poder determinar cuales documentos son relevantes para esa consulta es necesario establecer un orden de los mismo acorde a la importancia por lo tanto debemos convertir la consulta a un vector empleando el mismo proceso que vimos en el motor de búsqueda.

Se usa el mismo vocabulario $V$ empleado para la representación del corpus, y se procese a crear el vector $\overrightarrow{q_i}$, en caso que existan palabras en $q_i$ que no se encuentran en el vocabulario serán ignoradas. 

El valor que tomarán las componentes no tiene necesariamente que ser calculo igual a como se realiza en el motor de búsqueda para representar el corpus, puede depender de la función de ranking o si usamos un modelo un poco más complejo. En nuestro caso tomamos dos representaciones vectoriales acorde a como se calculaba el ranking.

\subsection*{Función de Ranking} 

Ahora es necesario relacionar la consulta con el corpus para determinar el valor de relevancia que tienen los documentos con respecto a la consulta realizada o sea debemos definir la función $R(q_i, d_j)$ para esto empleamos dos enfoques.

\paragraph{Puntuación Coincidente:} Una forma sencilla de calcular la similitud es emplear este enfoque primeramente representamos el vector de consulta $\overrightarrow{q_i}$ de forma binaria, asignando 1 para los términos que aparecen en la consulta, 0 en caso que no, luego para cada documento sumamos los valores de TF-IDF que tienen 1 en el vector consulta, de esta forma se obtiene el valor de similitud para realizar la ordenación.
Para realizar esta función solo es necesario hacer un producto escalar de los vectores

\begin{equation}
	R(q_i, d_j) = \overrightarrow{q_i} \cdot \overrightarrow{d_j} = \sum_{l = 0}^{|V|} w(t_l, d_j) \cdot w(t_l, q_i)
\end{equation}

De esta forma obtenemos mayores valores de relevancia para documentos que tengas valores de frecuencia alta para los términos que componen la consulta. Al representar el vector consulta solamente con 0 y 1 no aprovechamos la similitud que podría existir entre sus valores de frecuencia, o sea es posible que en la consulta podamos determinar que términos son más importantes que otros a la hora de realizar la búsqueda. 

\paragraph{Similitud del Coseno:} Primeramente colocaremos ahora el valor de TF-IDF asociado al término en la consulta y emplearemos el coseno del angulo comprendido entre los vectores consulta $\overrightarrow{q_i}$ y documento $\overrightarrow{d_j}$ para determinar la similitud, la función $R$ sería:

\begin{equation}
	R(q_i, d_j) = \frac{\overrightarrow{q_i} \cdot \overrightarrow{d_j}}{|\overrightarrow{q_i}| \cdot |\overrightarrow{d_j}|} = \frac{\sum_{l = 0}^{|V|} w(t_l, d_j) \cdot w(t_l, q_i)}{\sqrt{\sum_{l = 0}^{|V|} w^2(t_l, q_i)} \cdot \sqrt{\sum_{l = 0}^{|V|} w^2(t_l, d_j)}}
\end{equation}

Podemos apreciar como se busca establecer una relación entre los valores de frecuencia de la consulta y el documento de esta forma analizamos aquellos términos de la consulta que son más importantes y buscamos similitud con respecto al corpus.

\paragraph{Output del Sistema:} Una vez que tenemos la función de ranking pasamos a obtener los valores correspondientes para la consulta, ordenamos de forma descendente aquellos que son distintos de 0. Suministrando el titulo del documento y el valor obtenido.

\subsection*{Retroalimentación de Relevancia}

La retroalimentación de relevancia es una característica de algunos sistemas de recuperación de información . La idea detrás de la retroalimentación de relevancia es tomar los resultados que se devuelven inicialmente de una consulta determinada, recopilar la retroalimentación de los usuarios y utilizar información sobre si esos resultados son relevantes o no para realizar una nueva consulta. Podemos distinguir de manera útil entre tres tipos de retroalimentación: retroalimentación explícita, retroalimentación implícita y retroalimentación ciega o "pseudo".

Debido a que emplear el TF-IDF no siempre resulta en una representación confiable debido a que cuando representamos los vectores perdemos el significado semántico además de las relaciones entre los términos, por ello decidimos incluir un algoritmo de retroalimentación explícito en nuestro caso, con el objetivo de suministrar al usuario una manera de mejorar el vector consulta, así podemos resolver problemas de ambigüedad entre otros para esto elegimos el algoritmo de Rocchio el cual es abordado en [CITAR ROCCHIO].

\subsubsection*{Algoritmo de Rocchio:} Un algortimo simple y muy conocido para la retroalimentación en los sistemas de recuperación de información es el algoritmo de Rocchio, el cual era empleado en los IRS que surgieron del Sistema de Recuperación de Información SMART, desarrollado en 1960-1964. El enfoque de retroalimentación de Rocchio se desarrolló utilizando el Modelo de Espacio Vectorial que empleaba SMART. El algoritmo se basa en la suposición de que la mayoría de los usuarios tienen una concepción general de qué documentos deben indicarse como relevantes o no relevantes para la consulta que realizan. Por lo tanto, la consulta de búsqueda del usuario se revisa para incluir un porcentaje arbitrario de documentos relevantes y no relevantes como un medio para aumentar la precisión del motor de búsqueda.

El algoritmo de rocchio busca realizar una mejora a la consulta que deseamos, podemos ver como consulta $\overrightarrow{q_i}$ realiza una bipartición del conjunto $D$ en dos conjunto $D_R$ donde se encuentran aquellos documentos cuya relevancia es distinta de 0 y $D - D_R$ serán aquellos que no son relevantes. Ahora debemos encontrar una consulta optima a partir de $\overrightarrow{q_i}$ para la cual se maximice la similitud con el conjunto de documentos relevantes $D_R$ implicando un valor mínimo de similitud para el conjunto de textos no relevantes:

\begin{equation}
	\overrightarrow{{q_i}}_{opt} = \verb*|argmax|_{\overrightarrow{q_i}} \left[ R(\overrightarrow{q_i}, D_R) - R(\overrightarrow{q_i}, D - D_R) \right] 
\end{equation}

Podemos emplear como $R$ la similitud del coseno vista anteriormente o también el centroide del conjunto de documentos seleccionado, este último resulta más útil que ya que necesitamos reducir la similitud acorde a un subconjunto de $D$, aplicando el centroide y sustituyendo se obtiene:

\begin{equation}
	\overrightarrow{{q_i}}_{opt} = \frac{1}{|D_R|} \sum_{\overrightarrow{d_j} \in D_R} \overrightarrow{d_j} - \frac{1}{|D - D_R|} \sum_{\overrightarrow{d_j} \notin D_R} \overrightarrow{d_j} 
\end{equation}

Este enfoque no es muy utilizado frecuentemente debido a que el conjunto $D$ puede ser muy grande por lo tanto determinar el centroide sobre una bipartición para optimizar la consulta no resultará muy eficiente. Por eso se emplean 3 parámetros $\alpha, \beta $ y $\gamma$  para definir peso asociado a la consulta original, el conjunto de documentos relevantes y el de no relevantes, de esta forma podemos seleccionar subconjuntos. La expresión matemática final será:

\begin{equation}
	\overrightarrow{q_m} = \alpha · \overrightarrow{q_0} + \beta \frac{1}{|D_R|} \sum_{\overrightarrow{d_j} \in D_R} \overrightarrow{d_j} - \gamma \frac{1}{|D - D_R|} \sum_{\overrightarrow{d_j} \notin D_R} \overrightarrow{d_j}
\end{equation}

Donde $\overrightarrow{q_0}$ es el vector de la consulta que se desea optimizar, $D_R$ es el conjunto formado por los documentos relevantes, los parámetros funcionan como valores de confianza a la hora de operar con cada miembro de la expresión, por ejemplo si quisiéramos tener muchos documentos en ambas categorías entonces deberíamos dar valores mayores a $\beta$ y $\gamma$. Al aplicar esta expresión la consulta original se acerca más hacia el centroide de documentos relevantes y se aleja de los no relevantes, está nueva consulta será ahora la empleada por el IRS. Los valores negativos en el nuevo vector se asumen como 0, debido a que solamente son validos los valores para el primer cuadrante. Como hemos dicho esto mejora la precisión y el recall del sistema, principalmente en la práctica se ha demostrado como el recall es el más beneficiado ayudando a optimizar la búsqueda. Un aspecto importante es como la interacción con el usuario determina que tan buen valor de recall obtenemos, el tiempo que tomen en revisar los resultados, evaluarlos y determinar si satisface lo deseado es fundamental. En la práctica es común usar $\alpha = 1, \beta = 0.75$ y $\gamma = 0.15$, como se aprecia se busca mantener el sentido de la consulta original y obtener la mayor cantidad de documentos relevantes.

\section{Sobre la implementación} % aqui link a github

La implementación del sistema se realizó en \verb*|Python|, ya que cuenta con muchas herramientas para el procesamiento de texto, la representación de vectores de forma optima y al ser tan empleado para realizar trabajos con IRS se cuenta con una extensa documentación.

El objeto que representa el IRS es \verb*|IrSystem|, el cual contiene las implementaciones de los temas abordados en la sección anterior. Para crear una instancia se emplea el valor de $\alpha$ para definir el peso que se le da al título de los documentos, y la dirección del dataset que será empleado, el cual contiene el corpus. Si nuestro corpus no contiene textos con título solamente tendremos que hacer $\alpha = 0$, para que no se tome en cuenta el peso del título.

\noindent
%
\begin{verbatim}
	class IrSystem:
	   
	   def __init__(self, alpha, dataset):
	      ...
	      
	   def preprocess(self, data):
	     ...
	     
	   ...
\end{verbatim}
%
\noindent

Para realizar el preprocesado empleamos \textbf{nltk}, que es una poderosa herramienta en el procesamiento de lenguaje natural. Para retirar las stopwords se empleo \verb*|nltk.corpus.stopwords| que brinda una colección extensa de palabras en ingles. Para el proceso de stemming se usó \verb*|nltk.stem.PorterStemmer|, el PorterStemmer retira sufijos y prefijos de las palabras dejando solamente la raíz, es tal vez uno de los más empleados. Adicionalmente para tokenizar el texto preprocesado se empleo \verb*|nltk.tokenize.word_tokenize|.

Para el proceso de vectorización se emplean matrices esparcidas las cuales permiten representar vectores con 0 en muchas de sus componentes, para esto empleamos \verb*|scipy.sparse|. Para construir estos vectores primero se termino el vocabulario y luego se paso a implementar los cálculos de frecuencia. Para realizar calculos más comodos y poder manejar el conteo se empleo \verb*|numpy| y \verb*|collection.Counter|, de esta forma se crearon las funciones para calcular los pesos.



\section{Evaluación del Sistema}

\section{Resultados Obtenidos}

\section{Recomendaciones}

\begin{thebibliography}{4}

\bibitem{jour} Smith, T.F., Waterman, M.S.: Identification of Common Molecular
Subsequences. J. Mol. Biol. 147, 195--197 (1981)

\bibitem{lncschap} May, P., Ehrlich, H.C., Steinke, T.: ZIB Structure Prediction Pipeline:
Composing a Complex Biological Workflow through Web Services. In: Nagel,
W.E., Walter, W.V., Lehner, W. (eds.) Euro-Par 2006. LNCS, vol. 4128,
pp. 1148--1158. Springer, Heidelberg (2006)

\bibitem{book} Foster, I., Kesselman, C.: The Grid: Blueprint for a New Computing
Infrastructure. Morgan Kaufmann, San Francisco (1999)

\bibitem{proceeding1} Czajkowski, K., Fitzgerald, S., Foster, I., Kesselman, C.: Grid
Information Services for Distributed Resource Sharing. In: 10th IEEE
International Symposium on High Performance Distributed Computing, pp.
181--184. IEEE Press, New York (2001)

\bibitem{proceeding2} Foster, I., Kesselman, C., Nick, J., Tuecke, S.: The Physiology of the
Grid: an Open Grid Services Architecture for Distributed Systems
Integration. Technical report, Global Grid Forum (2002)

\bibitem{url} National Center for Biotechnology Information, \url{http://www.ncbi.nlm.nih.gov}

\end{thebibliography}

\end{document}
